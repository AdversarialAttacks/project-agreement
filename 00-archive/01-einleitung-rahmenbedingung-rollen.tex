\section{Einleitung}
Die vorliegende Projektvereinbarung beschreibt das Projekt \\ 
``24FS\_I4DS27 - Adversarial Attacks - Wie kann KI überlistet werden?". \\\\
In diesem Dokument werden die Ausgangslage, Ziele, Daten, Methodik, Pflichttermine, Zeitplan, Technologien und Ressourcen sowie Risiken, Methodik, Daten, Risiken und , Verantwortlichkeiten, die Projektplanung, der Umfang der Arbeit sowie die zentrale Fragestellung definiert. Es dient als verbindliche Grundlage für alle Projekt beteiligten, um die erfolgreiche Durchführung und Zielerreichung zu gewährleisten.

\section{Rahmenbedingungen}
Das Projekt erstreckt sich über einen Zeitraum von 26 Wochen vom \textbf{19.02.2024} bis zum \textbf{13.09.2024}. 

Für die erfolgreiche Durchführung dieser Arbeit erhält jeder Studierende \textbf{12 ECTS}, was einem Zeitaufwand von 360 Stunden pro Semester entspricht. 

\section{Rollen}
Die Rollenverteilung zwischen Studierenden und Fachbetreuern ist eine wichtige organisatorische Massnahme, um sicherzustellen, dass ein Projekt effektiv durchgeführt wird.

\begin{itemize}
    \item \textbf{Studierende} \\
    Die Studierenden sind die Hauptausführenden des Projekts. Sie sind dafür verantwortlich, die Projektarbeit durchzuführen und die zugewiesenen Aufgaben mit höchstem Engagement zu bearbeiten. 
    
    \item \textbf{Fachbetreuer} \\
    Die Fachbetreuer spielen eine unterstützende und lenkende Rolle im Projekt. Sie bieten Fachwissen, Erfahrung und Anleitung, um sicherzustellen, dass das Projekt erfolgreich verläuft. 
\end{itemize}