\section{Wissenschaftliches Vorgehensweise}

Um eine fundierte wissenschaftliche Grundlage für unser Projekt zu gewähr-leisten, verpflichten wir uns zu folgenden Prinzipien:

\begin{itemize}
    \item \textbf{Literaturüberprüfung} \\
    Durchführung einer umfassenden Überprüfung bestehender Forschungsarbeiten, um den aktuellen Wissensstand zu erfassen und unsere Forschungsfragen entsprechend einzuordnen.
    
    \item \textbf{Methodische Klarheit} \\ 
    Auswahl und präzise Definition der Forschungsmethoden basierend auf den spezifischen Anforderungen unserer Fragestellung, um Nachvollziehbarkeit und Reproduzierbarkeit zu sichern.
    
    \item \textbf{Kritische Bewertung} \\
    Unsere Ergebnisse werden kritisch im Kontext bestehender Forschung bewertet, Limitationen diskutiert und Implikationen für zukünftige Forschung erörtert.
    
    \item \textbf{Transparenz und Dokumentation} \\
    Detaillierte Dokumentation aller Forschungsphasen, einschliesslich Methoden, Datenquellen und Analyseverfahren, um Transparenz zu ge-währleisten und Überprüfbarkeit zu ermöglichen.
    
\end{itemize}

\section{Methode} %Technische Vorgehensweise
Bei der Projektplanung setzen wir auf eine agile Vorgehensweise mit 2-Wochen-Sprints. Am Ende jeder Sprintphase halten die Studierenden eine interne Sitzung ab, um die Aufgaben für den nächsten Sprint zu besprechen. Das Feedback der Fachbetreuer fliesst in die Aufgabenstellung ein.

Wöchentliche Meetings mit Fachexperten ermöglichen den Studierenden, ihren Fortschritt zu präsentieren und Feedback einzuholen. Die Frequenz dieser Treffen kann je nach Bedarf angepasst werden. Die wichtigsten Erkenntnisse und eine Traktandenliste werden von den Studierenden dokumentiert.

Das allgemeine Vorgehen der Bachelorarbeit folgt einem wissenschaftlichen Ansatz.

\section{Pflichttermine \& Meilensteine}
In diesem Kapitel dokumentieren wir die wichtigsten Pflichttermine \& Meilensteine.

\begin{table}[ht]
    \centering
    \begin{tabular}{@{}lp{8cm}@{}}
        \toprule
        \textbf{Pflichttermin} & \textbf{Datum / Hinweis} \\
        \midrule
        Projektvereinbarung & Bis zum 13.03.2024 beim Fachbetreuer einzureichen. \\
        \midrule
        Zwischenpräsentation & Wird in Absprache mit den Studierenden und dem Fachbetreuer terminiert. \\
        \midrule
        Schlusspräsentation & Wird in Absprache mit den Studierenden und dem Fachbetreuer terminiert. \\
        \midrule
        Ausstellung Bachelorthesen & Geplant für den 16.08.2024. \\
        \midrule
        Abgabe & Ebenfalls am 16.08.2024. \\
        \midrule
        Verteidigung & Findet zwischen dem 02.09.2024 und dem 13.09.2024 statt. \\
        \toprule
        \textbf{Meilenstein} & \textbf{Datum / Hinweis} \\
        \midrule
        Meilenstein 1  & blabla \\
        \midrule
        Meilenstein 1  & blabla \\
        \bottomrule
    \end{tabular}
\end{table}
