\section{Technologien und Ressourcen}
Unsere Code-Entwicklung basiert hauptsächlich auf Python unter Verwendung von Python-Skriptdateien und Jupyter Notebooks. Für Deep Learning-Projekte verwenden wir das PyTorch-Framework und überwachen und analysieren die Modelle mithilfe eines geeigneten Framework, wie bsp. Weights \& Bias. Wir nutzen auch KI-gestützte Tools wie ChatGPT und GitHub Copilot, um die Effizienz und Qualität unserer Arbeit zu verbessern.

Der Code, Dokumentationen und Entwicklungsprozesse werden in GitHub-Repositories verwaltet und mittels Git versioniert. Bei Bedarf an zusätzlicher Rechenkapazität für das Training der Modelle planen wir, auf die Ressourcen des CSCS oder i4Ds Slurm zurückzugreifen, die vom Data Science Studiengang oder vom i4Ds bereitgestellt werden.

Das Projektmanagement erfolgt innerhalb der GitHub-Organisation, während die Kommunikation über MS Teams abgewickelt wird.

\section{Risiken}
Das Ziel dieses Kapitels ist es, potenzielle Risiken zu identifizieren, zu bewerten und angemessene Massnahmen zu entwickeln, um deren Auswirkungen auf das Projekt zu minimieren.

\begin{itemize}
    \item \textbf{Studierenden-Ausfall} \\
    Bei längeren Ausfällen von Studierenden wird eine Neubewertung des Projekts mit den Fachbetreuern durchgeführt, um mögliche Anpassungen zu besprechen.
    
    \item \textbf{Fachbetreuer-Ausfall} \\
    Bei längeren Ausfällen eines Fachbetreuers organisiert der verbleibende Fachbetreuer einen adäquaten Ersatz, um die Projektkontinuität sicherzustellen.
    
    \item \textbf{These Risiko} \\
    Eine agile Vorgehensweise ermöglicht es, Risiken zu mindern, indem Anpassungen an der Arbeit vorgenommen werden, falls die These abgelehnt wird.

    \item \textbf{Fragestellung unbeweisbar} \\
    Sollte sich die Haupt- oder Unterfragen als unbeweisbar erweisen, dokumentieren wir die Gründe und differenzieren zwischen beantwortbaren und unbeantwortbaren Aspekten.

    \item \textbf{Ressourcen Kosten} \\
    Bei unvorhergesehenen Kosten wird eine gründliche Prüfung alternativer Lösungen durchgeführt, einschliesslich der Möglichkeit eines Antrags auf Kostenbeteiligung seitens der Hochschule.
    
\end{itemize}

\section{Lieferobjekte}
Als primäre Projektleistung ist ein Bericht zu erstellen, der die durchgeführten Projektarbeiten, Auswertungen und Erkenntnisse behandelt. \\\\
Darüber hinaus findet eine Zwischenpräsentation während des Semesters sowie eine Abschlusspräsentation und Verteidigung am Ende der Arbeit statt. Die Termine für die Präsentationen werden in gegenseitiger Absprache festgelegt. \\\\
Sämtlicher für die Aufgabenstellung relevanter Code und Notebooks werden in der GitHub Organisation im jeweiligen Repository abgelegt und dokumentiert und sind öffentlich einsehbar.




%------------------------------------------
%Welche Daten sollten wir verwenden
%Risiken, (Infrastruktur, Kosten, etc)