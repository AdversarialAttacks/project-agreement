\section{Einleitung}
Die vorliegende Projektvereinbarung beschreibt das Projekt \\ 
``24FS\_I4DS27 - Adversarial Attacks - Wie kann KI überlistet werden?". \\\\
In diesem Dokument werden die Ausgangslage, Ziele, Daten, Methodik, Pflichttermine, Zeitplan, Technologien und Ressourcen sowie Risiken, Methodik, Daten, Risiken und , Verantwortlichkeiten, die Projektplanung, der Umfang der Arbeit sowie die zentrale Fragestellung definiert. Es dient als verbindliche Grundlage für alle Projekt beteiligten, um die erfolgreiche Durchführung und Zielerreichung zu gewährleisten.

\section{Rahmenbedingungen}
Das Projekt erstreckt sich über einen Zeitraum von 26 Wochen vom \textbf{19.02.2024} bis zum \textbf{13.09.2024}. 

Für die erfolgreiche Durchführung dieser Arbeit erhält jeder Studierende \textbf{12 ECTS}, was einem Zeitaufwand von 360 Stunden pro Semester entspricht. 

\section{Rollen}
Die Rollenverteilung zwischen Studierenden und Fachbetreuern ist eine wichtige organisatorische Massnahme, um sicherzustellen, dass ein Projekt effektiv durchgeführt wird.

\begin{itemize}
    \item \textbf{Studierende} \\
    Die Studierenden sind die Hauptausführenden des Projekts. Sie sind dafür verantwortlich, die Projektarbeit durchzuführen und die zugewiesenen Aufgaben mit höchstem Engagement zu bearbeiten. 
    
    \item \textbf{Fachbetreuer} \\
    Die Fachbetreuer spielen eine unterstützende und lenkende Rolle im Projekt. Sie bieten Fachwissen, Erfahrung und Anleitung, um sicherzustellen, dass das Projekt erfolgreich verläuft. 
\end{itemize}

\newpage
\section{Ausgangslage}
\todo{Dafür brachen wir die Fragestellung}
\begin{itemize}
    \item Mehrwert des Projektes, wieso wichtig ?
    \item Stand der Forschung ?
\end{itemize}

\section{Ziel der Arbeit / Fragestellung}
\todo{Dafür brauchen wir die Fragestellung}
Was wollen wir mit dieser Arbeit herausfinden/erreichen?
Was gehört zu unseren Zielen, und was nicht?
Wie lautet unsere Fragestellung?

\section{Daten}
\todo{Dafür brauchen wir die Fragestellung}
Hier Daten beschreiben die wir nutzen.
Bilderdaten?
CIFAR10 ? CIFAR 100 ? ImageNet? 
https://pytorch.org/vision/stable/datasets.html

\section{Wissenschaftliches Vorgehensweise}

Um eine fundierte wissenschaftliche Grundlage für unser Projekt zu gewähr-leisten, verpflichten wir uns zu folgenden Prinzipien:

\begin{itemize}
    \item \textbf{Literaturüberprüfung} \\
    Durchführung einer umfassenden Überprüfung bestehender Forschungsarbeiten, um den aktuellen Wissensstand zu erfassen und unsere Forschungsfragen entsprechend einzuordnen.
    
    \item \textbf{Methodische Klarheit} \\ 
    Auswahl und präzise Definition der Forschungsmethoden basierend auf den spezifischen Anforderungen unserer Fragestellung, um Nachvollziehbarkeit und Reproduzierbarkeit zu sichern.
    
    \item \textbf{Kritische Bewertung} \\
    Unsere Ergebnisse werden kritisch im Kontext bestehender Forschung bewertet, Limitationen diskutiert und Implikationen für zukünftige Forschung erörtert.
    
    \item \textbf{Transparenz und Dokumentation} \\
    Detaillierte Dokumentation aller Forschungsphasen, einschliesslich Methoden, Datenquellen und Analyseverfahren, um Transparenz zu ge-währleisten und Überprüfbarkeit zu ermöglichen.
    
\end{itemize}

\section{Technische Vorgehensweise} %Methode

Bei der Projektplanung setzen wir auf eine agile Vorgehensweise, bei der wir 2-Wochen-Sprints durchführen. Am Ende jeder Sprintphase halten die Studierenden eine interne Sitzung ab, in der besprochen wird, welche Aufgaben im nächsten Sprint anstehen. Bei der Erstellung der Aufgaben wird auch das Feedback der Fachbetreuer einbezogen.

Wöchentlich findet ein Meeting mit den Fachexperten statt, bei dem die Studierenden ihre Fortschritte präsentieren und Feedback einholen. Die Frequenz der Meetings kann je nach Bedarf angepasst werden. Die wichtigsten Erkenntnisse und eine Traktandenliste der Meetings zwischen Fachexperten und Studierenden werden von den Studierenden dokumentiert.

Das Allgemeine Vorgehen der Bachelorarbeit ist ein wissenschaftliches Vorgehen

\section{Pflichttermine \& Meilensteine}
In diesem Kapitel dokumentieren wir die wichtigsten Pflichttermine.

\begin{itemize}
    \item \textbf{Projektvereinbarung} \\
    Die Projektvereinbarung ist bis zum \textbf{13.03.2024} beim Fachbetreuer einzureichen.
    
    \item \textbf{Zwischenpräsentation} \\
    Die Zwischenpräsentation wird in Absprache mit den Studierenden und dem Fachbetreuer terminiert.
    
    \item \textbf{Schlusspräsentation} \\
    Die Schlusspräsentation wird in Absprache mit den Studierenden und dem Fachbetreuer terminiert. 

    \item \textbf{Ausstellung Bachelorthesen} \\
    Die Ausstellung der Bachelorarbeiten ist für den \textbf{16.08.2024} geplant.
    
    \item \textbf{Abgabe} \\
    Die Abgabe der Bachelorarbeit erfolgt ebenfalls am \textbf{16.08.2024}.

    \item \textbf{Verteidigung} \\
    Die Verteidigung findet zwischen dem \textbf{02.09.2024} und dem \textbf{13.09.2024} statt.
    
\end{itemize}

Die Meilensteine zur Beurteilung des Projektfortschritts sind wie folgt definiert.

\todo{Dafür brauchen wir die Fragestellung}

\begin{itemize}
    \item \textbf{}
\end{itemize}

\section{Technologien und Ressourcen}
Die Entwicklung unseres Codes erfolgt primär unter Verwendung von Python, wobei sowohl Python-Skriptdateien als auch Jupyter Notebooks zum Einsatz kommen. Für die Umsetzung von Deep Learning-Projekten bedienen wir uns des PyTorch-Frameworks. Die Überwachung und Analyse der Modelle wird durch den Einsatz von Weights \& Bias ermöglicht. Zusätzlich unterstützen wir unseren Entwicklungsprozess durch den Gebrauch von KI-gestützten Tools wie ChatGPT und GitHub Copilot, um Effizienz und Qualität unserer Arbeit zu steigern.

Die Entwicklung, Monitoring und Aktualisierung von Code sowie Dokumentationen erfolgen über die Repositories innerhalb der entsprechenden GitHub-Organisation und werden mittels git versioniert.

Im Falle, dass das Training der Deep Learning-Modelle mehr Rechenkapazität verlangt, als unsere aktuelle Hardware bereitstellen kann, planen wir, auf die Ressourcen des CSCS oder I4DS Slurm zurückzugreifen. Diese Ressourcen werden vom Data Science Studiengang oder vom i4DS zur Verfügung gestellt.

Das Projektmanagement wird innerhalb der GitHub-Organisation durchgeführt, während die Kommunikation über MS-Teams sichergestellt wird.

\newpage
\section{Risiken}
Das Ziel dieses Kapitels ist es, potenzielle Risiken zu identifizieren, zu bewerten und angemessene Massnahmen zu entwickeln, um deren Auswirkungen auf das Projekt zu minimieren.

\begin{itemize}
    \item \textbf{Studierenden-Ausfall} \\
    Sollte ein Studierender über einen längeren Zeitraum ausfallen und dadurch wichtige Fragestellungen nicht mehr beantworten können, wird eine gemeinsame Neubewertung des Projekts mit den Fachbetreuern durchgeführt, um mögliche Anpassungen zu erörtern.
    
    \item \textbf{Fachbetreuer-Ausfall} \\
    Im Falle des längeren Ausfalls eines Fachbetreuers, der das Projekt gefährden könnte, obliegt es dem verbleibenden Fachbetreuer, einen adäquaten Ersatz zu organisieren, um die Kontinuität des Projekts sicherzustellen.
    
    \item \textbf{These Risiko} \\
    Im Falle einer potenziellen Ablehnung unserer These könnten die Relevanz unserer Unterfragen beeinträchtigt werden. Eine agile Vorgehensweise ermöglicht es jedoch, Risiken zu mindern, indem Anpassungen an unserer Arbeit vorgenommen werden können.

    \item \textbf{Fragestellung unbeweisbar} \\
    Sollte sich die Haupt- oder Unterfragen als unbeweisbar erweisen, dokumentieren wir die Gründe und differenzieren zwischen beantwortbaren und unbeantwortbaren Aspekten.

    \item \textbf{Ressourcen Kosten} \\
    Sollten unvorhergesehene Kosten auftreten, die allein nicht bewältigt werden können, wird eine gründliche Prüfung alternativer Lösungen durchgeführt, einschliesslich der Möglichkeit eines Antrags auf Kostenbeteiligung seitens der Hochschule.
    
\end{itemize}
\newpage
\section{Lieferobjekte}
Als primäre Projektleistung ist ein Bericht zu erstellen, der die durchgeführten Projektarbeiten, Auswertungen und Erkenntnisse behandelt. \\\\
Darüber hinaus findet eine Zwischenpräsentation während des Semesters sowie eine Abschlusspräsentation und Verteidigung am Ende der Arbeit statt. Die Termine für die Präsentationen werden in gegenseitiger Absprache festgelegt. \\\\
Sämtlicher für die Aufgabenstellung relevanter Code und Notebooks werden in der GitHub Organisation im jeweiligen Repository abgelegt und dokumentiert und sind öffentlich einsehbar.

\newpage
\section{Vereinbarung}
Diese Projektvereinbarung wird von beiden Parteien akzeptiert und bildet die Grundlage für die Zusammenarbeit im Rahmen des Projekts \\``24FS\_I4DS27 - Adversarial Attacks". Änderungen an dieser Vereinbarung erfordern die Zustimmung beider Parteien.

\vspace{1.5cm}

\noindent\begin{tabular}{ll}
\makebox[2.5in]{\hrulefill} & \makebox[2.5in]{\hrulefill}\\
Daniel Perruchoud & Datum \\[1.5cm]

\makebox[2.5in]{\hrulefill} & \makebox[2.5in]{\hrulefill}\\
Stephan Heule & Datum\\[1.5cm]

\makebox[2.5in]{\hrulefill} & \makebox[2.5in]{\hrulefill}\\
Si Ben Tran & Datum\\[1.5cm]

\makebox[2.5in]{\hrulefill} & \makebox[2.5in]{\hrulefill}\\
Gabriel Torres Gamez & Datum\\[1.5cm]

\end{tabular}


%------------------------------------------
%Welche Daten sollten wir verwenden
%Risiken, (Infrastruktur, Kosten, etc)