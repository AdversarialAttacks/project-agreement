
\section{Risiken}
\todo{Nochmals anschauen nach Fragestellungwahl}
Das Ziel dieses Kapitels ist es, potenzielle Risiken zu identifizieren, zu bewerten und angemessene Massnahmen zu entwickeln, um deren Auswirkungen auf das Projekt zu minimieren.

\begin{itemize}
    \item \textbf{These Risiko} \\
    Eine agile Vorgehensweise ermöglicht es, Risiken zu mindern, indem Anpassungen an der Arbeit vorgenommen werden, falls die These abgelehnt wird.

    \item \textbf{Fragestellung unbeweisbar} \\
    Sollte sich die Haupt- oder Unterfragen als unbeweisbar erweisen, dokumentieren wir die Gründe und differenzieren zwischen beantwortbaren und unbeantwortbaren Aspekten.

    \item \textbf{Ressourcen Kosten} \\
    Bei unvorhergesehenen Kosten wird eine gründliche Prüfung alternativer Lösungen durchgeführt, einschliesslich der Möglichkeit eines Antrags auf Kostenbeteiligung seitens der Hochschule.
    
\end{itemize}