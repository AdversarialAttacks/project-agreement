\section{Risiken}
Das Ziel dieses Kapitels ist es, potenzielle Risiken zu identifizieren, zu bewerten und angemessene Massnahmen zu entwickeln, um deren Auswirkungen auf das Projekt zu minimieren.

\begin{enumerate}
    \item \textbf{Unzureichende Wirksamkeit von Abwehrstrategien} \\ 
    Es besteht das Risiko, dass die entwickelten Abwehrmechanismen gegen universelle adversarial Perturbationen nicht die erwartete Effektivität zeigen. Sollten die Abwehrstrategien in der Praxis weniger wirksam sein als angenommen, könnte dies darauf hindeuten, dass die Grundannahme der Fragestellung - nämlich dass Krankheitserkennungssysteme effektiv gegen solche Bedrohungen geschützt werden können – nicht haltbar ist.

    \item \textbf{Datenqualität} \\
    Die Qualität und Diversität der verwendeten Datensätze sind entscheidend. Unvollständige oder verzerrte Datensätze können zu einer schlechten Generalisierung des Modells führen.
    
    \item \textbf{Ressourcenmanagement} \\
    Die Verfügbarkeit von Rechenkapazitäten ist begrenzt. Technische Einschränkungen könnten das Projekt beeinträchtigen und Meilensteine unhaltbar machen.

\end{enumerate}

In der folgenden Tabelle schätzen wir die Eintrittswahrscheinlichkeit P (1-10) und den Schweregrad S (1-10) der Risiken ein. Ein Risikowert wird berechnet R, indem die Werte der Eintrittswahrscheinlichkeit und Schweregrad zusammen multipliziert werden. 

\begin{table}[ht]
    \centering
    \begin{tabular}{@{}p{0.25cm}p{7cm}p{1cm}p{1cm}p{1cm}@{}}
        \toprule
        \textbf{\#} & \textbf{Risikoart} & \textbf{P} & \textbf{S} & \textbf{R} \\
        \midrule
        1 & Unzureichende Wirksamkeit von Abwehrstrategien & 5 & 3 & \colorbox{green}{15} \\
        \midrule
        2 & Datenqualität & 2 & 5 & \colorbox{green}{10} \\
        \midrule
        3 & Ressourcenmanagement & 8 & 4 & \colorbox{yellow}{32} \\
        \bottomrule
    \end{tabular}
    \caption{Risikoanalyse}
\end{table}


